
\startsetups[doc:classify]
	\hfill
	\cap \WORD \getvariable{env}{classify}
	\hfill
\stopsetups

\startsetups[doc:reminder]
	\getmarking[chapternumber]
	\hskip.25em
	\getmarking[chapter]
\stopsetups

\startsetups[doc:revision]
	\small
	Rev.~\getvariable{env}{revision}
\stopsetups

% disabled, as we place this by \setupfootertexts
\setuppagenumbering[location=]

% grid={line,-25pt},

\setuplabeltext[appendix=Appendix~]

\definestructureconversionset[frontpart:pagenumber][][romannumerals]
\definestructureconversionset[bodypart:pagenumber][][numbers]

\startsectionblockenvironment[frontpart]
\stopsectionblockenvironment

\startsectionblockenvironment[bodypart,appendix]
\stopsectionblockenvironment

\def\TitleRev#1%
	{#1\hfill \setup{doc:revision}}

\define[2]\HeadVbox%
	{\vbox to 1.25\lineheight % XXX: fragile
		{\headtext{chapter} #1 \hskip.75em #2}}

\setuphead[chapter,appendix,section,subsection][
	align={flushleft, nothyphenated, verytolerant},
	incrementnumber=yes]

\setuphead[title][
	page=yes,
	header=high,
	textcommand=\TitleRev,
	after={\blank[2*halfline]}]

% header=high to have the chapter title occupy the space of the header.
% To keep the geometric layout, the title height must match the header height.
\setuphead[chapter,appendix][
	page=yes,
	header=high,
	after=\placecontent,
	placehead=yes,
	command=\HeadVbox,
	style=\bfc]

\setuphead[section,subsection][
	page=no,
	before={\blank[2em]},
	style=\rm\tfb]


\startmode[manpage]

	\setuphead[section][
		after=,
		aftersection=]

\stopmode

